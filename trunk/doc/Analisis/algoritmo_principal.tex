\documentclass[a4paper,10pt]{article}
\usepackage[spanish]{babel}
\usepackage{amssymb}
\usepackage[utf8]{inputenc}
\usepackage[left=2.5cm,top=3.5cm,right=2.5cm,bottom=3.5cm]{geometry}
\usepackage{caption}
\usepackage{graphicx}
\usepackage{float}

\title{Diseño de Videojuegos \\ Algoritmo Principal}
\author{Rosa María Durante Lerate\\Pablo Recio Quijano\\Noelia Sales Montes}

\begin{document}
\maketitle

\noindent Se ejecuta el programa. \\
Se cargan los ficheros de configuración. \\
Se inicia el videojuego mostrando el menú principal.

\begin{itemize}
\item El jugador elige jugar.\\
Se muestra en pantalla las opciones de crear nuevo juego o cargar uno existente.

\begin{itemize}
\item El jugador crea un nuevo juego. \\
Se cargan los ficheros iniciales. \\
Se muestra
  una animación de la introducción del juego.

\item El jugador elige  cargar un juego existente. \\
Se cargan los ficheros configurados
  para el usuario.
\end{itemize}

El usuario comienza a jugar (con el motor de movimiento).\\ 
El jugador mueve el personaje.

\begin{itemize}
\item El personaje se mueve sin ningún inconveniente.
\item El personaje no se puede mover (colisiona con un objeto).
\item El personaje no se puede mover pero si interactuar:

\begin{itemize}
\item Comienza un combate.

\item Interactua con un objeto.

\item Dialoga con otro personaje.  

\end{itemize}

\end{itemize}

El jugador entra en el inventario.

\begin{itemize}
\item Visualiza la pestaña contenedora del estado de los personajes.
\item Visualiza la pestaña contenedora de los objetos que tiene el grupo.
\item Visualiza la pestaña que contiene las opciónes Guardar y Salir.

\begin{itemize}
\item El usuario guarda el juego. \\ El sistema guarda el estado del juego.
\item El usuario sale del juego. \\ El sistema destruye la memoria dinámica utilizada y
  sale del programa.

\end{itemize}
\end{itemize}

\item El jugador elige salir. \\ El sistema sale del programa.
\end{itemize}

\clearpage

\paragraph*{Combate}
\noindent El sistema carga la apariencia y los ficheros necesarios. \\
El sistema calcula los turnos en el combate. \\
Mientras que uno de los dos grupos no muera o nuestro grupo no escape:

\begin{itemize}
\item Indica el tipo de ataque que vaya a realizar.
\item Se realiza.
\item Se calcula el daño causado decrementando la vida del personaje que lo recibió y
  calcula el número de puntos especiales (si es el caso) y se decrementa.
\item Comprobamos el estado de los equimos.
\end{itemize}

\begin{itemize}
\item Si nuestro equipo ha ganado:
\begin{itemize}
\item Se calcula y añade la experiencia obtenida. Si algún personaje alcanza el número de
  puntos de experiencia necesarios para pasar de nivel se notifica al usuario y se
  redefinen  las características de dicho personaje.
\end{itemize}

\item Si nuestro equipo ha muerto:
\begin{itemize}
\item El sistema da la posibilidad al usuario de volver a combatir o de volver al estado
  de antes de comenzar el combate, por si quisiera evitarlo. Cuando sea un combate de
  historia sólo podrá combatir de nuevo, y no evitarlo.
\end{itemize}

\item Si nuestro equipo huye:
\begin{itemize}
\item Vuelve al estado anterior al combate.
\end{itemize}

\end{itemize}


\paragraph*{Interacción con un objeto}

\noindent El sistema muestra la acción que puede realizar con el objeto.
\begin{itemize}
\item El personaje realiza dicha acción. El objeto se guarda en el inventario y se quita
  de la pantalla. (Es en este momento cuando ya se puede atravesar).
\item El personaje no realiza ninguna acción. El objeto queda en pantalla para que pueda
  realizar la acción más tarde (si el jugador quisiera). 
\end{itemize}

\paragraph*{Diálogo con un personaje}
El sistema indica el diálogo con un personaje no jugable.
\begin{itemize}
\item Al final del diálogo entra en combate con el personaje no jugable.
\item Al final del diálogo regala un objeto a nuestro personaje.
\end{itemize}

\end{document}

