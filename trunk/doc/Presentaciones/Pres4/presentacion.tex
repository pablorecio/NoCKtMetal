% ===================================================================
%                   Presentación con Latex Beamer
% ===================================================================
\documentclass[9pt,xcolor=svgnames]{beamer}
%\documentclass[handout,xcolor=svgnames]{beamer} %Version imprimible
% -------------------------------------------------------------------
% Paquetes personalizados
\usepackage{../paquetes}
\usepackage{../colores}
\usepackage{../info}
\usepackage{../modo}
% -------------------------------------------------------------------

% Comienza el documento
\begin{document}
% Tikz -> Imágenes
\tikzstyle{every picture}+=[remember picture]
% Entorno matemático
\everymath{\displaystyle}
% -------------------------------------------------------------------
\subtitle{La venganza del proyecto}
\date{Junio 2009}
% -------------------------------------------------------------------
% Fondo blanco: primera página
% ------------------------------------------------------------------

\beamersetaveragebackground{white}

\begin{frame}
 \thispagestyle{empty}
 \begin{figure}[t]
  \centering
  \includegraphics[scale=0.4]{./Imagenes/logo_cachimba.pdf}

\noindent \Huge Presenta...
\end{figure}
\end{frame}


\begin{frame}
 \thispagestyle{empty}
 
 \animate<2-3> 
 \begin{figure}[t]
  \centering
  \includegraphics<1>[scale=0.7]{../Imagenes/logo_1.pdf}
  \includegraphics<2>[scale=0.7]{../Imagenes/logo_2.pdf}
  \includegraphics<3>[scale=0.7]{../Imagenes/logo_3.pdf}
  \includegraphics<4>[scale=0.7]{../Imagenes/logo_4.pdf}
 \end{figure}
\end{frame}


% -------------------------------------------------------------------
% Fondo para el resto del documento
\setbeamertemplate{background}{
\includegraphics[width=\paperwidth,height=\paperheight]
{../Imagenes/fondo.pdf}
}
% -------------------------------------------------------------------


% Continuación:
% Transparencia de Inicio -> Título
\begin{frame}
 \titlepage
\end{frame}

\normalsize

% Transparencia de índice
\begin{frame}{Índice} 
 \transdissolve
 \tableofcontents
\end{frame}
  
  
 \section{Estado final del proyecto}
 
   
   \subsection{Modificaciones sufridas por el proyecto}
   
   \begin{frame}{Modificaciones y justificaciones}
   \transdissolve
    
    \begin{itemize}
     \item Desde la anterior presentación no hemos considerado
	   ninguna modificación del proyecto inicial, por lo
	   estamos siguiendo el plan inicial.
	 
    \end{itemize}    
    
   \end{frame}
   
 \section{Contenido extra del proyecto}

  \subsection{Documentación}

   \begin{frame}{Manual de instalación y uso}
    \transdissolve
    
    Se incluye un manual de instalación y uso del videojuego
   \end{frame}


   \begin{frame}{Documentación}
    \transdissolve
    
    Existe documentación interna (automatizada con Doxygen o similar) y
    se muestra un ejemplo de documento de ingeniería.

   \end{frame}


   \subsection{Elementos multimedia}

   \begin{frame}{Estado actual: Motor de Combate}
   \transdissolve

    Se comenta en los créditos del juego y la documentación el origen de
    los elementos visuales y sonoros del juego y existe algo hecho por
    ellos.\\

    Captura de pantalla de la pantalla de créditos que muestra lo
    solicitado.
   \end{frame}



  \subsection{Posibilidad de ampliación}


   \begin{frame}{Documentación}
    Es posible ampliar el juego (nuevos niveles, enemigos, etc) y se
    documenta cómo:

    \begin{enumerate}
     \item Diseño del nuevo nivel: mapa, PJs, objetos, desarrollo de la
	   historia...
     \item Desarrollo de los sprites de movimiento y de combate de los
	   PJ a añadir, así como de los nuevos elementos multimedia, si
	   los hubiere.
     \item Traducción a XML siguiendo las pautas dadas en la
	   documentación pertinente de las estadísticas de los nuevos
	   PJ, del mapa, y de las posiciones de cada elemento y su
	   comportamiento en el mapa. Posterior encriptación
     \item Inclusión de los ficheros multimedia en los directorios
	   pertinentes (también se encuentra en la documentación de
	   ayuda al futuro desarrollador), así como de los ficheros XML.
     \item Y, por último, probar que todo va perfecto... A JUGAR!
    \end{enumerate}
   \end{frame}

  \subsection{Elementos innovadores}

   \begin{frame}{Documentación}
    El juego incorpora elementos innovadores:
    \begin{itemize}
     \item Los combates son completamente ``libres'' desde el punto de
	   vista de la acción del usuario: puede golpear, curar y hacer
	   la acción que desee (siempre que posea los requisitos
	   mínimos) sobre cualquier PJ, sea o no sea del propio equipo.\\

     \item (Los PJs a veces tienen comportamientos en mitad del combate,
	   fruto de su personalidad:\\
	   Por ejemplo, Baldos arrea al enanín para rematarlo una de
	   tantas veces, como muestra de su gran fuerza ante el equipo
	   rival (aunque a Graimito no le haga mucha gracia...) )

     \item (Inclusión de la utilización de libSDL en PSP: adaptación del
	   juego a PSP, o preparación para ello - aún sin desarrollar
	   por completo)

    \end{itemize}
   \end{frame}

 \section{Presentación del producto}

  \subsection{¡A jugar!}
 \begin{frame}{Estado del menu principal}
   \transdissolve

  Captura de pantalla del juego en funcionamiento.\\

  Paso a la ejecución del juego.
 \end{frame}
 

  
 \section{Para terminar}

  
\scriptsize

 \begin{frame}{Esta presentación es libre}
  Copyright 2009 Noelia Sales Montes\\
  Parte del Proyecto NoCKt Metal\\
  \url{http://nocktmetal.forja.rediris.es/}
  
  \vspace*{0.3cm}
  
  Creative Commons Attribution License
  \url{http://creativecommons.org/licenses/by/2.0/}\\
  Creative Commons, 559 Nathan Abbott Way, Stanford, California 94305,
  USA.
  
  \vspace*{0.3cm}
  
  Este trabajo se publica bajo la siguiente licencia:\\
  Creative Commons Attribution License
  \url{http://creativecommons.org/licenses/by/3.0/}
  
  \vspace*{0.3cm}
  
  Usted es libre de:
  \begin{itemize}
   \item copiar, distribuir y comunicar públicamente la obra
   \item hacer obras derivadas
  \end{itemize}
  
  Bajo las condiciones siguientes:
  \begin{itemize}
   \item Reconocimiento. Debe reconocer los créditos de la obra de la
	 manera especificada por el autor o el licenciador (pero no de una
	 manera que sugiera que tiene su apoyo o apoyan el uso que hace de
	 su obra).
   \item Al reutilizar o distribuir la obra, tiene que dejar bien claro los
	 términos de la licencia de esta obra.
   \item Alguna de estas condiciones puede no aplicarse si se obtiene el
	 permiso del titular de los derechos de autor
   \item Nada en esta licencia menoscaba o restringe los derechos morales
	 del autor.
  \end{itemize}
 \end{frame}


  
\end{document}
 
