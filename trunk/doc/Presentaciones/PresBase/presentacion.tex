% ===================================================================
%                   Presentación con Latex Beamer
% ===================================================================
\documentclass[9pt,xcolor=svgnames]{beamer}
%\documentclass[handout,xcolor=svgnames]{beamer} %Version imprimible
% -------------------------------------------------------------------
% Paquetes personalizados
\usepackage{../paquetes}
\usepackage{../colores}
\usepackage{../info}
\usepackage{../modo}
% -------------------------------------------------------------------

% Comienza el documento
\begin{document}
% Tikz -> Imágenes
\tikzstyle{every picture}+=[remember picture]
% Entorno matemático
\everymath{\displaystyle}
% -------------------------------------------------------------------

% -------------------------------------------------------------------
% Fondo blanco: primera página
\beamersetaveragebackground{white}

\begin{frame}
 \thispagestyle{empty}

 \animate<2-3> 
 \begin{figure}[t]
  \centering
  \includegraphics<1>[scale=0.7]{../Imagenes/logo_1.pdf}
  \includegraphics<2>[scale=0.7]{../Imagenes/logo_2.pdf}
  \includegraphics<3>[scale=0.7]{../Imagenes/logo_3.pdf}
  \includegraphics<4>[scale=0.7]{../Imagenes/logo_4.pdf}
 \end{figure}
\end{frame}
% -------------------------------------------------------------------

% -------------------------------------------------------------------
% Fondo para el resto del documento
\setbeamertemplate{background}{
\includegraphics[width=\paperwidth,height=\paperheight]
{../Imagenes/fondo.pdf}
}
% -------------------------------------------------------------------


% Continuación:
% Transparencia de Inicio -> Título
\begin{frame}
 \titlepage
\end{frame}

\normalsize

% Transparencia de índice
\begin{frame}
 \frametitle{Índice} 
 \transboxin
 \tableofcontents
\end{frame}
  
  
 \section{Guía de estilo}

 \begin{frame}
  \frametitle{Plantilla}

  \transdissolve
   Tras la distribución de secciones a cada colaborador, se proporciona una
   plantilla a los colaboradores.
  
  \begin{block}{Características}
    \begin{itemize}
    \item Bastante extensa.
    \item Ejemplos variados.
    \item Nuevo estilo derivado de la plantilla dada por el profesor.    
    \end{itemize}
  \end{block}
  
  \begin{block}{Objetivos}
  \begin{itemize}
    \item Simplificar el trabajo de los colaboradores.
    \item Reducir el tiempo que se necesita para desarrollar el tema a 
    gusto de cada uno.  
  \end{itemize}
  \end{block}
 
  \end{frame}
   
 
 \section{Guía de estilo y formato}
 
 \begin{frame}
  \frametitle{Formatos}

  \transdissolve
  
  \noindent  Se subió al foro junto con la plantilla una guía de estilo, en 
  la cual se indicaba también una nueva posibilidad.
  
  \begin{block}{Entrega de diversos formatos}
  \begin{itemize}
    \item El formato decidido por el coordinador era PPT
    \item La revisora emplea OOov3.0 $\Longrightarrow$ ODP
    \item Conclusión: Posibilidad abierta
  \end{itemize}
  \end{block}
  
  \begin{block}{Entregas realizadas}
  \begin{itemize}
    \item 9 entregas en PPT
    \item 4 entragas en ODP\footnote{Aunque tan sólo se hubiera entregado
    un trabajo en ODP, ya habría valido la pena abrir la posibilidad.}
  \end{itemize}
  \end{block} 
 
  \end{frame}
  
  
  \section{Redistribución de las secciones}
    \transboxin
    
  \begin{frame}
   \frametitle{Redistribución de secciones}
 
    \begin{block}{Pautas del profesor}
     El trabajo final que organizará el Revisor lo ordenará de forma que las
     estadísticas sean los primeros trabajos, la evolución de los 
     computadores españoles vengan a continuación y en la última parte 
     estén las características de los demás computadores.
    \end{block}
  
  \begin{block}{Solución}
   La distribución de los temas dada por el coordinador no se vió variada, 
   pero hubo que modificar el orden en que estos temas aparecen.
  
   Esto tampoco influyó en la asignación de trabajos:
   \textsc{tan sólo se vió modificado el número de sección de cada 
   colaborador}.
  
   \begin{itemize}
     \item No se tuvo en cuenta el no modificar el número de sección.
     \item La mayoría de los colaboradores lo corrigieron.
   \end{itemize}
  \end{block}
  
  \end{frame}
 
  
  \section{Revisiones}
  
  \begin{frame}
  \frametitle{Errores comunes}
    \transdissolve
    
  \textsc{No ha habido errores conceptuales}
  
  En general, tan sólo ha habido pequeños errores de estilo o de 
  transcripción:
  
  \begin{itemize}
  \item Confusión de código de secciones.
  \item Faltas ortográficas leves.
  \item Descuadre entre versiones (lógico)
  \end{itemize}
  
  Todos se corrigieron en la revisión final (espero)
  
  \end{frame}
  
  \section{Agradecimientos}
  
  \begin{frame}
  \frametitle{Agradecimientos}
    \transdissolve
    
  \begin{itemize}
  \item Prontitud
  \item Interés
  \item Colaboración
  \end{itemize}
  
  \vspace*{1cm}
  
  A todos los colaboradores: enhorabuena por el trabajo realizado

  \end{frame}
  
 \end{document}
