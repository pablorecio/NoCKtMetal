% ===================================================================
%                   Presentación con Latex Beamer
% ===================================================================
\documentclass[9pt,xcolor=svgnames]{beamer}
%\documentclass[handout,xcolor=svgnames]{beamer} %Version imprimible
% -------------------------------------------------------------------
% Paquetes personalizados
\usepackage{../paquetes}
\usepackage{../colores}
\usepackage{../info}
\usepackage{../modo}
% -------------------------------------------------------------------

% Comienza el documento
\begin{document}
% Tikz -> Imágenes
\tikzstyle{every picture}+=[remember picture]
% Entorno matemático
\everymath{\displaystyle}
% -------------------------------------------------------------------
\subtitle{El retorno del proyecto}
\date{Mayo 2009}
% -------------------------------------------------------------------
% Fondo blanco: primera página
% ------------------------------------------------------------------

\beamersetaveragebackground{white}

\begin{frame}
 \thispagestyle{empty}
 \begin{figure}[t]
  \centering
  \includegraphics[scale=0.4]{./Imagenes/logo_cachimba.pdf}

\noindent \Huge Presenta...
\end{figure}
\end{frame}


\begin{frame}
 \thispagestyle{empty}
 
 \animate<2-3> 
 \begin{figure}[t]
  \centering
  \includegraphics<1>[scale=0.7]{../Imagenes/logo_1.pdf}
  \includegraphics<2>[scale=0.7]{../Imagenes/logo_2.pdf}
  \includegraphics<3>[scale=0.7]{../Imagenes/logo_3.pdf}
  \includegraphics<4>[scale=0.7]{../Imagenes/logo_4.pdf}
 \end{figure}
\end{frame}


% -------------------------------------------------------------------
% Fondo para el resto del documento
\setbeamertemplate{background}{
\includegraphics[width=\paperwidth,height=\paperheight]
{../Imagenes/fondo.pdf}
}
% -------------------------------------------------------------------


% Continuación:
% Transparencia de Inicio -> Título
\begin{frame}
 \titlepage
\end{frame}

\normalsize

% Transparencia de índice
\begin{frame}{Índice} 
 \transdissolve
 \tableofcontents
\end{frame}
  
  
 \section{Estado del proyecto}
 
  \subsection{Actualización de la planificación}
  
  \begin{frame}{Diagrama de Gannt}
   
%   \begin{figure}[t]
%    \includegraphics[scale=0.27]{./Imagenes/gannt.pdf}
%   \end{figure}
   
  \end{frame}
  
  
  \begin{frame}{Asignación de Recursos}
   
   Seguimos centrándonos en la implementación:
   
    
  \end{frame}
   
   
   
   \subsection{Modificaciones sufridas por el proyecto}
   
   \begin{frame}{Modificaciones y justificaciones}
    
    \begin{itemize}
     \item Desde la anterior presentación no hemos considerado
	   ninguna modificación del proyecto inicial, por lo
	   estamos siguiendo el plan inicial.
	 
     \end{itemize}
    
    
   \end{frame}
   
   
 \section{Presentación Previa del producto}
 
   \subsection{Producto actual}

   %INCLUIR EXPLICACION DE LA IA
   \begin{frame}{Estado actual: Motor de Combate}
    \begin{itemize}
     \item Inclusión de una simple IA para los rivales
     \item Mejora de la serialización de los objetos, añadiendo
	   construcciones desde ficheros.
     \item Adquisición de experiencia de los PJs
     \item Más grupos enemigos con distintas cualidades cada uno.
     \item Clase \texttt{Biblioteca} que indexa los datos para 
	   guardados de partida.
     \item Solucionados algunos bugs críticos
    \end{itemize}
   \end{frame}

   \begin{frame}{Captura de pantalla: Motor de combate}
    
   \end{frame}

   \begin{frame}{Estado actual: Motor de movimiento}
    \begin{itemize}
     \item Añadidos NPJs en el motor, ampliando la funcionalidad del mismo.
     \item Añadidas las funcionalidades de diálogos entre PJs y NPJs.
     \item Tratamiento de eventos para futura interaccion entre PJs y
	   NPJs u objetos
     \item Mejora de las colisiones en el mapa, detectando además los NPJ
     \item Solucionados algunos bugs y re-escritura de algunas clases
	   para una mejor cohesión
     \item Menu inicial :)
    \end{itemize}
   \end{frame}


   \begin{frame}{Captura de pantalla: Motor gráfico}
    
   \end{frame}



 \section{Calidad de la implementación}
 
  \subsection{Calidad del codigo}
  
  \begin{frame}{Programación Orientada a Objetos + SDL}   
   
   
   \textsc{Nota.} La unión entre los dos motores aún no está
   implementada, puesto que no están terminados los motores.

  \end{frame}
  
  
  \subsection{Ejemplos de código}
  
  \begin{frame}{Ejemplo: Clase x}
   combatiente.h recortado xD
  \end{frame}
  
  
  \begin{frame}{Ejemplo: Interfaz principal entre clases}
   personaje.h
  \end{frame}
    
  


 \section{Conclusión}
 
  \subsection{Solución de problemas}

  \begin{frame}{Problemas y soluciones}
   - Ordenacion del vector en los turnos
   - Utilizacion de vectores de bajo nivel fallaba varias
     veces. Solucion: std::vector<> 
   - IA totalmente aleatoria: ponderacion por porcentaje de vida.
   - Fechas en la que nos encontramos xD: solucion retrasar plazos
   - Problema en capas SDL\_Surface. Solucion: redibujado de algunas
     capas en una sola
   - Aceleracion del menu inicial debido a un redibujado de los
     elementos del mismo. Solucion: dibujar los elementos una unica vez

  
  \end{frame}
  
  
 \section{Para terminar}
  
\scriptsize

 \begin{frame}{Esta presentación es libre}
  Copyright 2009 Noelia Sales Montes\\
  Parte del Proyecto NoCKt Metal\\
  \url{http://nocktmetal.forja.rediris.es/}
  
  \vspace*{0.3cm}
  
  Creative Commons Attribution License
  \url{http://creativecommons.org/licenses/by/2.0/}\\
  Creative Commons, 559 Nathan Abbott Way, Stanford, California 94305,
  USA.
  
  \vspace*{0.3cm}
  
  Este trabajo se publica bajo la siguiente licencia:\\
  Creative Commons Attribution License
  \url{http://creativecommons.org/licenses/by/3.0/}
  
  \vspace*{0.3cm}
  
  Usted es libre de:
  \begin{itemize}
   \item copiar, distribuir y comunicar públicamente la obra
   \item hacer obras derivadas
  \end{itemize}
  
  Bajo las condiciones siguientes:
  \begin{itemize}
   \item Reconocimiento. Debe reconocer los créditos de la obra de la
	 manera especificada por el autor o el licenciador (pero no de una
	 manera que sugiera que tiene su apoyo o apoyan el uso que hace de
	 su obra).
   \item Al reutilizar o distribuir la obra, tiene que dejar bien claro los
	 términos de la licencia de esta obra.
   \item Alguna de estas condiciones puede no aplicarse si se obtiene el
	 permiso del titular de los derechos de autor
   \item Nada en esta licencia menoscaba o restringe los derechos morales
	 del autor.
  \end{itemize}
 \end{frame}


  
\end{document}
 
