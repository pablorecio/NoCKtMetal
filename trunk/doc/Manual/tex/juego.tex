\section{Prefacio}

No One Can Kill The Metal es el nombre completo de este videojuego que nos 
sumerge en la historia de una banda nacida como las mejores: en un garage 
cualquiera.

Con un estilo que nos recuerda a un clásico RPG (Role-Playing Game) iremos 
avanzando en las aventuras de esta banda mientras nos ponemos en su piel 
para conseguir su objetivo: ¡triunfar! ... o al menos ganar algun que otro 
concurso.

Después de esto... \textbf{¿no tienes ganas de empezar a jugar?}

\subsection{Transfondo}

En este juego te metes en la piel de la banda \textbf{Ampli Breakers}, una
pequeña banda Heavy con unos grandes sueños. Esta banda descubre un concurso
con el cual puede ganar un contrato con una empresa discográfica, y deciden
participar. \\

Pero no todo iba a ser fácil...

\section{Jugando}

Para comenzar a jugar, es tan simple como arrancar el ejecutable 
\texttt{nocktmetal}, generado en el proceso de compilación.

Una vez dentro del menu, puedes elegir la opción de ``Nueva partida'', 
``Cartar partida'' ó ``Salir''.

\subsection{Controles}

Las siguientes teclas son las predeterminadas, y no está previsto la opción
de poder cambiarlas.

\begin{description}
\item[ESC] Salir del juego, salir del menú
\item[Flecha arriba] Movimiento hacia arriba/elección del elemento superior 
  al actual del menú en que nos encontremos
\item[Flecha abajo] Movimiento hacia abajo/elección del elemento inferior 
  al actual del menú en que nos encontremos
\item[Flecha izquierda] Movimiento hacia la izquierda/elección del elemento 
  a la izquierda del actual del menú en que nos encontremos
\item[Flecha derecha]	Movimiento hacia la derecha/elección del elemento a 
  la derecha al actual del menú en que nos encontremos
\item[Barra espaciadora] Menú principal del juego/Acceso al inventario
\item[Return]	Aceptar/Entrar en subsección del menú/Interactar con NPJ 
  u objeto
\end{description}

\subsection{Miembros de la banda}

Como la mayoría de las bandas, tenemos 4 componentes:

\begin{description}
\item[Manolo la Punki: ] La cantante del grupo. Está bastante equilibrada, no
  destaca demasiado en nada.
\item[Dentacos Joe: ] Dentacos es el guitarrista de la banda. Es muy rápido 
  y tiene una virtuosidad bastante alta, pero no es muy fuerte.
\item[Baldos: ] El batería de los Ampli Breakers. Antes derribaba árboles
  a golpe de baqueta, así que es bastante fuerte, pero no demasiado listo
  ni rápido.
\item[Graimito el Bajo: ] El marginado de la banda: el bajista. Graimito es
  muy debil y fragil, pero es el curandero de la banda, así que teneis
  que protejerlo.
\end{description}
