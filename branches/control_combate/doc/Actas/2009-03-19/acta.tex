\documentclass[a4paper,10pt]{article}
\usepackage[utf8]{inputenc}
\usepackage[spanish]{babel}

\usepackage[left=2.5cm,top=3.5cm,right=2.5cm,bottom=3.5cm]{geometry}

\usepackage{array}

\usepackage{graphicx} 

\begin{document}

\includegraphics[scale=0.6]{../uca.png}

\begin{center}
  \noindent \huge\textbf{ Acta de reunión de grupo}\\
\end{center}

\noindent Día 19 de Marzo del 2009\\

\noindent Asisten:

\begin{center}
\begin{tabular}{|m{5.2cm}|m{5.2cm}|m{5.2cm}|}
  \hline
  Rosa María Durante Lerate & Pablo Recio Quijano & Noelia Sales Montes\\
  & & \\[2cm]
  Fdo: & Fdo: & Fdo: \\
  \hline
\end{tabular}
\end{center}

\noindent Acuerdos:\\

\noindent Primero, hemos identificado que el juego va a tener 2
subsistemas independientes, aunque trabajando bajo la misma BD
probablemente. Estos sistemas serán:

\begin{description}
\item[Movimiento: ] Vista a priori cenital, con el movimiento del
  personaje principal por el mundo, de un punto a otro del mapa,
  obteniendo objetos y desarrollando la historia.
\item[Combate: ] El sistema de combate será totalmente independiente;
  otro tipo de gráficos y de reglas de movimiento.
\end{description}

\noindent Aun así, todavía no es definitivo que sea un RPG o un Action
RPG, dada la mayor simplicidad de este último.\\

\noindent Acordados el nombre definitivo del proyecto: \textbf{NoCKt
  Metal}, del grupo protagonista: \textbf{The Ampli Breakers} y de los
4 personajes:
\begin{itemize}
\item \textbf{Manololapunki}. Cantante
\item \textbf{Dentacos Joe}. Guitarra
\item \textbf{Baldos}. Bateria
\item \textbf{Graimito el Bajo}. Bajista
\end{itemize}

\noindent Tomadas decisiones sobre la historia principal, aunque no
son reflejadas en este acta. Aún falta una necesidad de concretar qué
herramientas vamos utilizar, el soporte para la colaboración, y muchas
más cosas.

\end{document}
