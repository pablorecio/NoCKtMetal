\documentclass[a4paper,10pt]{article}
\usepackage[utf8]{inputenc}
\usepackage[spanish]{babel}

\usepackage[left=2.5cm,top=3.5cm,right=2.5cm,bottom=3.5cm]{geometry}

\usepackage{array}

\usepackage{graphicx} 

\begin{document}

\includegraphics[scale=0.6]{../uca.png}

\begin{center}
  \noindent \huge\textbf{ Acta de reunión de grupo}\\
\end{center}

\noindent Día 20 de Marzo del 2009\\

\noindent Asisten:

\begin{center}
\begin{tabular}{|m{5.2cm}|m{5.2cm}|m{5.2cm}|}
  \hline
  Rosa María Durante Lerate & Pablo Recio Quijano & Noelia Sales Montes\\
  & & \\
  & & \\
  & & \\
  & & \\
  & & \\
  & & \\
  & & \\
  Fdo: & Fdo: & Fdo: \\
  \hline
\end{tabular}
\end{center}

\noindent Acuerdos:\\

\noindent La reunión tubo los siguientes puntos del dia:

\begin{itemize}
\item Concretar estilo de juego
\item Definir funcionalidades del motor (motores) del juego
\item Concretar diseño de los personajes
\item Primera aproximación al diagrama ERe de la Base de Datos
\end{itemize}

\noindent El primer punto surge a raiz de la conversación con Alejo
Acevedo y David Ballesteros, ya que comenzamos a plantearnos la
posibilidad de modificar algo el enfoque del juego, y en lugar
realizar un jRPG, hacer un Action RPG (estilo Zelda), de forma que
solo tubieramos que implementar uno de los dos motores del juego.\\

\noindent Sin embargo, pensamos que se podrían reutilizar clases de
ambos motores, por lo que continuamos con la idea del jRPG al estilo
FF, Chrono Trigger y demás clásicos.\\

\noindent Con respecto al segundo punto, definimos totalmente las
acciones de los dos motores del juego, sin embargo no están plasmados
en este acta, pues no es el cometido de la misma. Sin embargo, se debe
hacer una excepción y explicar muy brevemente los sistemas de
juego. 

\noindent Nuestro juego va a estar dividido en dos motores o entornos:
uno será el denominado de 'movimiento' ya que en este entorno el
usuario tendrá el manejo del personaje para moverlo por el mapa, bajo
una vista cenital.

\noindent El segundo sistema o motor, denominado 'de batalla', permitirá
al jugador enfrentarse con los enemigos que se oportunen en un entorno
de visión lateral. \\

\noindent El tercer punto viene a raiz de si realizar personajes
humanoides, o abstacciones pelotudas de dichos humanoides para
simplificación de la parte gráfica del juego. \\

\noindent El último punto del día hace referencia a la primera
aproximación de la base de datos que debe sostener el juego. La
intención es tener claro qué entidades pertenecerán al sistema y qué
requisistos (en conocimiento sobre el juego) necesitamos para
completar dicha base de datos. \\

\end{document}
