% ===================================================================
%                   Presentación con Latex Beamer
% ===================================================================
\documentclass[9pt,xcolor=svgnames]{beamer}
%\documentclass[handout,xcolor=svgnames]{beamer} %Version imprimible
% -------------------------------------------------------------------
% Paquetes personalizados
\usepackage{../paquetes}
\usepackage{../colores}
\usepackage{../info}
\usepackage{../modo}
% -------------------------------------------------------------------

% Comienza el documento
\begin{document}
% Tikz -> Imágenes
\tikzstyle{every picture}+=[remember picture]
% Entorno matemático
\everymath{\displaystyle}
% -------------------------------------------------------------------
\subtitle{La venganza del proyecto}
\date{Junio 2009}
% -------------------------------------------------------------------
% Fondo blanco: primera página
% ------------------------------------------------------------------

\beamersetaveragebackground{white}

\begin{frame}
 \thispagestyle{empty}
 \begin{figure}[t]
  \centering
  \includegraphics[scale=0.4]{./Imagenes/logo_cachimba.pdf}

\noindent \Huge Presenta...
\end{figure}
\end{frame}


\begin{frame}
 \thispagestyle{empty}
 
 \animate<2-3> 
 \begin{figure}[t]
  \centering
  \includegraphics<1>[scale=0.7]{../Imagenes/logo_1.pdf}
  \includegraphics<2>[scale=0.7]{../Imagenes/logo_2.pdf}
  \includegraphics<3>[scale=0.7]{../Imagenes/logo_3.pdf}
  \includegraphics<4>[scale=0.7]{../Imagenes/logo_4.pdf}
 \end{figure}
\end{frame}


% -------------------------------------------------------------------
% Fondo para el resto del documento
\setbeamertemplate{background}{
\includegraphics[width=\paperwidth,height=\paperheight]
{../Imagenes/fondo.pdf}
}
% -------------------------------------------------------------------


% Continuación:
% Transparencia de Inicio -> Título
\begin{frame}
 \titlepage
\end{frame}

\normalsize

% Transparencia de índice
\begin{frame}{Índice} 
 \transdissolve
 \tableofcontents
\end{frame}
  
  
 \section{Estado final del proyecto}
 
   
   \subsection{Modificaciones sufridas por el proyecto}
   
   \begin{frame}{Modificaciones y justificaciones}
   \transdissolve
    
    \begin{itemize}
     \item Desde la anterior presentación no hemos considerado
	   ninguna modificación del proyecto inicial, por lo
	   que estamos siguiendo el plan inicial.
	 
    \end{itemize}    
    
   \end{frame}
   
 \section{Contenido extra del proyecto}

  \subsection{Documentación}

   \begin{frame}{Manual de instalación y uso}
    \transdissolve
    
    Se incluye un manual de instalación y uso en
    \textsc{html}\footnote{Se puede acceder a través de la web del
    proyecto.} y \LaTeX.
    

    \begin{figure}[t]
     \centering
     \includegraphics[scale=0.18]{./Imagenes/manual.pdf}
    \end{figure}
   \end{frame}


   \begin{frame}{Documentación}
    \transdissolve
    
    La documentación interna está automatizada con Doxygen.\\

    Se puede encontrar accediendo desde la web del proyecto o
    directamente en el enlace
    \url{http://nocktmetal.forja.rediris.es/doc/index.html}.

   \end{frame}


   \subsection{Elementos multimedia}

   \begin{frame}{Créditos del juego}
   \transdissolve

    \begin{figure}[t]
     \centering
     \includegraphics[scale=0.3]{./Imagenes/creditos.pdf}
    \end{figure}
   \end{frame}



  \subsection{Posibilidad de ampliación}


   \begin{frame}{Documentación}
    Es posible ampliar el juego (nuevos niveles, enemigos, etc) y se
    documenta cómo:

    \begin{enumerate}
     \item Diseño del nuevo nivel: mapa, PJs, objetos, desarrollo de la
	   historia...
     \item Desarrollo de los sprites de movimiento y de combate de los
	   PJ a añadir, así como de los nuevos elementos multimedia, si
	   los hubiere.
     \item Traducción a XML siguiendo las pautas dadas en la
	   documentación pertinente de las estadísticas de los nuevos
	   PJ, del mapa, y de las posiciones de cada elemento y su
	   comportamiento en el mapa. Posterior encriptación
     \item Inclusión de los ficheros multimedia en los directorios
	   pertinentes (también se encuentra en la documentación de
	   ayuda al futuro desarrollador), así como de los ficheros XML.
     \item Y, por último, probar que todo va perfecto... A JUGAR!
    \end{enumerate}
   \end{frame}

  \subsection{Elementos innovadores}

   \begin{frame}{Documentación}
    El juego incorpora elementos innovadores:
    \begin{itemize}
     \item Los combates son completamente ``libres'' desde el punto de
	   vista de la acción del usuario: puede golpear, curar y hacer
	   la acción que desee (siempre que posean los requisitos
	   mínimos) sobre cualquier PJ, sea o no sea del propio equipo.\\

     \item Los PJs a veces tienen comportamientos en mitad del combate,
	   fruto de su personalidad:\\
	   Por ejemplo, Baldos arrea al enanín para rematarlo una de
	   tantas veces, como muestra de su gran fuerza ante el equipo
	   rival (aunque a Graimito no le haga mucha gracia...)\\

     \item Inclusión de la utilización de libSDL en PSP:
	   adaptación del juego a PSP, o preparación para ello - aún sin
	   desarrollar por completo.

    \end{itemize}
   \end{frame}

   \subsection{Dificultades en la implementación}

   \begin{frame}{Problemas encontrados (I)}
   El desarrollo ha sido de un nivel de programación algo complejo, ya que
   nos hemos encontrado muchos problemas en los dos motores:

   \begin{block}{Motor de movimiento}
    Hemos intentado siempre crear clases \textit{para
    todo}\footnote{Lógicamente, no a diestro y siniestro: siempre
    controlando que fueran de utilidad.}, abstrayendo lo más posible
    los datos de la SDL, lo cual no siempre ha sido fácil, porque queríamos
    que la librería fuera transparente para el programador y no le fuera
    necesario conocer como funciona SDL.\\
    
    \vspace*{0.3cm}
    Conseguir tener varios objetos de clases distintas ``colaborando'' de
    alguna manera dentro de una misma entidad de tipo ``mapa de tiles'', nos
    ha llevado varias horas de peleas con el compilador y entre nosotros.
   \end{block}
   
   \end{frame}

   \begin{frame}[fragile=singleslide]{Problemas encontrados (II)}
     \begin{block}{Motor de combate}
       En la implementación del motor de combate se intentó evitar desde un
       principio la redundancia de datos y objetos, para que al reescribir
       las cosas no hubiera inconsisencia.\\
       
      \vspace*{0.3cm}
       \textsc{¿Cómo se consiguió?} \textbf{Todo} en el motor de combate
      son punteros y referencias. Muy bonito, precioso todo, hasta que
      te equivocas en un sitio y se caen todos los punteros.\\
      Un ejemplo de instrucción sencilla puede ser:
\begin{verbatim}
_comb->_g1->getCombatientes().at(i)
     ->tiradaSuerte();
\end{verbatim}
       
       El control del turno tiene un puntero al control del combate, el cual
       tiene un puntero a cada grupo del combate. Estos grupos a su vez tienen
       un vector de punteros a combatiente. Más o menos, la
      implementación es así.
     \end{block}
   \end{frame}

   \begin{frame}{Problemas encontrados (y III)}
     Sin entrar en los motores en sí, también ha resultado complicado:
     \begin{itemize}
     \item El aprendizaje y uso \textbf{correcto} de la libSDL
     \item Las estadísticas de un juego de rol, cuyo control no es ni
	   mucho menos sencillo
     \item Comunicación y compenetración entre distintos programadores
     \item Para juegos jRPG no hay nada en la comunidad, por tanto
       hemos partido desde 0 prácticamente, lo cual no es lo mismo
       que partir con algo ya hecho de otros proyectos.
     \end{itemize}

    \begin{description}
     \item[Aclaración.] Lo que nos ha faltado por desarrollar no ha 
     sido por complicación, si no por falta de tiempo en general.
    \end{description}
   \end{frame}
 
 \section{Trabajo individual}

 \subsection{División básica}
 \begin{frame}{Divisón del trabajo}
   \transdissolve
   
   Desde el principio, intentamos dividir el proyecto en partes perfectamente
   distinguibles para repartirnos mejor el trabajo.\\

  \vspace*{0.3cm}
   La división ya se ha mencionado anteriormente. Optamos por dividir el
   juego en dos \textbf{motores} ó gestores, uno de movimiento y otro de
   combate, ya que tienen características muy diferentes.\\

  \vspace*{0.3cm}
  A continuación, especificaremos qué ha realizado cada componente del
  grupo de forma individual, con respecto al desarrollo principal del
  proyecto\footnote{No se han incluido aspectos más externos, como el
  desarrollo de la web, la búsqueda general de información/contenidos o
  el desarrollo del manual de usuario.}.

 \end{frame}

\subsection{Trabajo realizado por Rosa}
\begin{frame}{Rosa ha realizado...}
  \transdissolve
  
  \begin{itemize}
  \item Análisis y diseño del motor de movimiento, el cual se ha cumplido
    en un 80\% aproximadamente.
  \item Creación de las clases que trabajan con el mapa mediante tiles.
  \item Compenetración entre el mapa y los personajes.
  \item Ampliación de la clase \texttt{Animacion}.
  \item Inclusión de los NPJs en el motor de movimiento.
  \item Interacción del PJ con los elementos del escenario.
  \item Interacción entre ambos motores.
  \end{itemize}
\end{frame}

\subsection{Trabajo realizado por Pablo}
\begin{frame}{Pablo ha realizado}
  \transdissolve

  \begin{itemize}
  \item Análisis y diseño del motor de combate, el cual apenas se usó, por falta
    de perspectiva inicial.
  \item Pequeño estudio matemático para las estadísticas del combate, de
    forma que fueran compensadas.
  \item Todo el nucleo del combate, con sus clases e interacción entre ellas.
  \item Corrección de bugs.
  \item Mezclado de algunas músicas y efectos de sonido.
  \item Persistencia de los objetos para grabar partidas, usando XML.
  \item Un poco del guión.
  \end{itemize}
\end{frame}

\subsection{Trabajo realizado por Noelia}
\begin{frame}{Noelia ha realizado}
  \transdissolve

  \begin{itemize}
  \item Diseño gráfico general (sprites, tiles, elementos del menú,
	fondos).
  \item Movimiento de los personajes por pixels, compenetrado con el
	movimiento en tiles.
  \item Base de los PJs y NPJs mediante la clase \texttt{Personaje}.
  \item Abstracción general de los dos motores (funciones de llamada
	principales).
  \item Menú inicial.
  \item Abstracción de la inicialización de los sistemas, las pantallas
	y los eventos.
  \item Comienzo de la inclusión de gráficos al motor de combate.
  \end{itemize}
\end{frame}

\section{Presentación del producto}

  \subsection{¡A jugar!}

 \begin{frame}{Estado del menu principal}
   \transdissolve

  \begin{figure}[t]
   \centering
   \includegraphics[scale=0.3]{./Imagenes/movimiento.pdf}
  \end{figure}

 \end{frame}


 \begin{frame}{Transición de movimiento a combate}
  \transdissolve

  \begin{figure}[t]
   \centering
   \includegraphics[scale=0.28]{./Imagenes/combate.pdf}
  \end{figure}

 \end{frame}
 
 \section{Para terminar}
 \begin{frame}{Conclusiones}
   \transdissolve
   \begin{itemize}
   \item Diseñar y programar un videojuego es bastante distinto a lo que se
     imagina uno; sin embargo, hemos disfrutado dándonos de cabezazos con
     el compilador\footnote{En unos momentos más que en otros...}.
   \item Un juego no se puede hacer solo con programadores, es necesario un
	 equipo multidisciplinar: diseñadores, guionistas... Y una misma
	 persona ``no se puede'' dedicar a varios aspectos.
   \item Aunque el juego no está completo, realmente lo que hemos hecho 
     (estamos haciendo) es una librería para juegos RPG partiendo desde
     cero, cosa que no ha sido nada fácil.
   \end{itemize}
 \end{frame}

 \begin{frame}{Agradecimientos}
   \transdissolve
   \begin{itemize}
   \item Nos gustaría auto-felicitarnos porque, aunque el producto no está 
     completo, consideramos que hemos hecho un buen trabajo, partiendo de
     la base de que no había \textbf{nada} de este estilo bajo licencia libre.
   \item También felicitar a todos los grupos y compañeros que han trabajado
     y han llevado su videojuego adelante, la mayoría con muy buen resultado.
   \item Sin ánimo de ser \textit{pelotas}, agradecer a Manuel la dinámica de
     la asignatura, con la que hemos aprendido bastante a llevar (ó como no
     llevar) un proyecto en grupo, así como hacer algo más tangible que
     en otras asignaturas.
   \item Y aunque no es alumno de la asignatura, a David Casas Molina, que ha
     sido nuestro \textit{beta-tester} particular.
   \end{itemize}
 \end{frame}

 \begin{frame}{Herramientos utilizadas en el proyecto}
   \transdissolve
   \begin{description}
   \item[Subversion: ] Imprescindible algún sistema de control de versiones.
     Nos hubiera gustado usar Git, pero redIris trabaja con SVN.
   \item[C++: ] Todo el código está programado en C++ siguiendo el estándar ISO.
   \item[libSDL: ] Librería de abstracción multimedia.
   \item[Doxygen: ] Herramienta de documentación automática, para poder 
     generar una documentación interna de altísima calidad.
   \item[\LaTeX : ] Para las presentaciones y el manual de usuario.
   \item[Gimp: ] Para la creación y edición de sprites y tiles.
   \item[Inkscape: ] Editor de gráficos vectoriales, usado para la
	      edición de elementos diversos (menú).
   \item[LMMS: ] Usado para la generación de la música y efectos sonoros.
   \end{description}
 \end{frame}
 
  
\scriptsize

 \begin{frame}{Esta presentación es libre}
  Copyright 2009 Noelia Sales Montes\\
  Parte del Proyecto NoCKt Metal\\
  \url{http://nocktmetal.forja.rediris.es/}
  
  \vspace*{0.3cm}
  
  Creative Commons Attribution License
  \url{http://creativecommons.org/licenses/by/2.0/}\\
  Creative Commons, 559 Nathan Abbott Way, Stanford, California 94305,
  USA.
  
  \vspace*{0.3cm}
  
  Este trabajo se publica bajo la siguiente licencia:\\
  Creative Commons Attribution License
  \url{http://creativecommons.org/licenses/by/3.0/}
  
  \vspace*{0.3cm}
  
  Usted es libre de:
  \begin{itemize}
   \item copiar, distribuir y comunicar públicamente la obra
   \item hacer obras derivadas
  \end{itemize}
  
  Bajo las condiciones siguientes:
  \begin{itemize}
   \item Reconocimiento. Debe reconocer los créditos de la obra de la
	 manera especificada por el autor o el licenciador (pero no de una
	 manera que sugiera que tiene su apoyo o apoyan el uso que hace de
	 su obra).
   \item Al reutilizar o distribuir la obra, tiene que dejar bien claro los
	 términos de la licencia de esta obra.
   \item Alguna de estas condiciones puede no aplicarse si se obtiene el
	 permiso del titular de los derechos de autor
   \item Nada en esta licencia menoscaba o restringe los derechos morales
	 del autor.
  \end{itemize}
 \end{frame}


  
\end{document}
 
