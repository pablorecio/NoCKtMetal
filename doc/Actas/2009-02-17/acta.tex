\documentclass[a4paper,10pt]{article}
\usepackage[utf8]{inputenc}
\usepackage[spanish]{babel}

\usepackage[left=2.5cm,top=3.5cm,right=2.5cm,bottom=3.5cm]{geometry}

\usepackage{array}

\usepackage{graphicx} 

\begin{document}

\includegraphics[scale=0.6]{../uca.png}

\begin{center}
  \noindent \huge\textbf{ Acta de reunión de grupo}\\
\end{center}

\noindent Día 17 de Febrero de 2009\\

\noindent Asisten:

\begin{center}
\begin{tabular}{|m{5.2cm}|m{5.2cm}|m{5.2cm}|}
  \hline
  Rosa María Durante Lerate & Pablo Recio Quijano & Noelia Sales Montes\\
  & & \\
  & & \\
  & & \\
  & & \\
  & & \\
  & & \\
  & & \\
  Fdo: & Fdo: & Fdo: \\
  \hline
\end{tabular}
\end{center}

\noindent Acuerdos:\\

\begin{itemize}
\item Trás un estudio de las distintas posibilidades, hemos concretado
  que el proyecto ha realizar será un videojuego del género
  \textbf{RPG}. Esta elección se debe fundamentalmente a gustos
  personales además de como un reto para el grupo, ya que es bastante
  completo.\\

  \noindent Esta elección se baremó junto con la posibilidad de
  realizar una Aventura Gráfica, sin embargo, analizando a
  \textit{grosso modo}  ambos géneros, concretamos que RPG era más
  completo y mas interesante para realizar.
\item En una primera aproximación, hemos intentado ver las partes
  fundamentales que tendrá el software, cara a una futura modulación,
  y comenzar a abordar el análisis del sistema en función de
  esto. \\

  \noindent Hemos detectado a primera vista las siguientes partes:
  
  \begin{description}
  \item[Motor gráfico: ] La parte fundamental del sistema. Abstraerá
    la tecnología que usemos de forma que proporcione las
    funcionalidades directamente para este tipo de juegos.
  \item[Estadísticas: ] Las reglas del juego, respecto a puntos de
    habilidad, daños, defensa y demás.
  \item[Base de Datos: ] Será necesario tener una base de datos con
    los PJ, PNJ, los objetos, equipamiento...
  \item[Inteligencia Artificial: ] Dotar a los enemigos de cierta
    dificultad para batirlos.
  \end{description}


\item Claramente lo más importante es el motor del juego, y
  pensándolo más detenidamente, el objetivo será realizar unas
  librerias que permitan realizar un juego tipo RPG a partir de esta
  librería. Esta librería / motor tendrá las siguientes
  funcionalidades:

  \begin{itemize}
  \item Carga e interacción de escenarios, PJs, PNJs...
  \item Acciones tanto de movimiento en escenario como en batalla.
  \item IA del sistema.
  \item Sincronización con la BD.
  \item Manejo y tratamiento de estadísticas del juego.
  \item Diálogos con los PNJs.
  \item Realización de misiones.
  \end{itemize}
  
\item Por último, decidimos en común que lo más apropiado será usar un
  modelo de ciclo de vida incremental, de forma que vayamos poco a
  poco desarrollando el sistema, ya que tampoco podemos tener claros
  todos los requisitos del sistema, y usar un modelo secuencial, puede
  inducirnos a un fallo que en una fase de desarrollo avanzada puede
  ser fatal.
  
\item Aunque la idea inicial es desarrollar un motor genérico, en
  principio el videojuego que haremos para ``probar'' dicho motor,
  será un juego en el que el jugador maneje a un grupo heavy en una
  psicodélica aventura.
\end{itemize}

\end{document}