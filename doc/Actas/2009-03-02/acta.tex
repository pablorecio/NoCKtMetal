\documentclass[a4paper,10pt]{article}
\usepackage[utf8]{inputenc}
\usepackage[spanish]{babel}

\usepackage[left=2.5cm,top=3.5cm,right=2.5cm,bottom=3.5cm]{geometry}

\usepackage{array}

\usepackage{graphicx} 

\begin{document}

\includegraphics[scale=0.6]{../uca.png}

\begin{center}
  \noindent \huge\textbf{ Acta de reunión de grupo}\\
\end{center}

\noindent Día 3 de Marzo de 2009\\

\noindent Asisten:

\begin{center}
\begin{tabular}{|m{5.2cm}|m{5.2cm}|m{5.2cm}|}
  \hline
  Rosa María Durante Lerate & Pablo Recio Quijano & Noelia Sales Montes\\
  & & \\[2cm]
  Fdo: & Fdo: & Fdo: \\
  \hline
\end{tabular}
\end{center}

\noindent Acuerdos:\\

\noindent El objetivo de esta reunión es muy simple: identificar las
tareas preeliminares necesarias para la realización del proyecto,
analizar la necesidad de realizarlas los 3 o si son algo más concretas,
y repartir las primeras tareas.\\

\noindent Dichas tareas son:

\begin{enumerate}
\item Ampliación de MySQL. Teniendo en cuenta que nuestros
  conocimientos de MySQL son algo escasos para el proyecto, sobre todo
  en modificar y añadir datos. Además, es necesario ampliar los
  conocimientos sobre la API para C de MySQL.
\item Requisitos iniciales del sistema, tomados de juegos como los
  primeros Final Fantasy, Chrono Trigger y demás similares,
  básicamente tipo JRPG.
\item Localizar algunos motores para juegos, y ver ``por donde van los
  tiros''.
\item Comenzar a trabajar con la SDL para conocerla poco a poco.
\item Realización de las reglas del juego para los combates, las
  estadísticas, niveles, experiencia, etc...
\end{enumerate}

\noindent Estimando el tiempoe necesario para desarrollar cada una de
las tareas, pensamos que las tareas 1 y 5 requieren solo un
miembro, la tarea 2 necesita al menos de 2 miembros y las tareas 3 y 4
necesita que la realizemos todos. La distribución de trabajos es
la siguiente:

\begin{enumerate}
\item Pablo
\item Rosa y Noelia
\item Los 3
\item Los 3
\item Pablo, con posible apoyo del resto.
\end{enumerate}

\noindent La realización de dichas tareas han sido marcadas para
hacerlas en el plazo límite del viernes \textbf{20 de Marzo}, con la
idea de realizar reuniones y asignaciones de las tareas cada dos
semanas, para llevar un control paulatino del progreso del proyecto.

\end{document}
