\documentclass[a4paper,10pt]{article}
\usepackage[utf8]{inputenc}
\usepackage[spanish]{babel}

\usepackage[left=2.5cm,top=3.5cm,right=2.5cm,bottom=3.5cm]{geometry}

\usepackage{array}

\usepackage{graphicx} 

\begin{document}

\includegraphics[scale=0.6]{../uca.png}

\begin{center}
  \noindent \huge\textbf{ Acta de reunión de grupo}\\
\end{center}

\noindent Día 28 de Abril del 2009\\

\noindent Asisten:

\begin{center}
\begin{tabular}{|m{5.2cm}|m{5.2cm}|m{5.2cm}|}
  \hline
  Rosa María Durante Lerate & Pablo Recio Quijano & Noelia Sales Montes\\
  & & \\
  & & \\
  & & \\
  & & \\
  & & \\
  & & \\
  & & \\
  Fdo: & Fdo: & Fdo: \\
  \hline
\end{tabular}
\end{center}

\noindent Acuerdos:\\

\noindent La reunión viene a raiz de la necesidad de concretar las actuaciones
a realizar hasta la siguiente presentación intermedia, con fecha aproximada
del 20 de mayo. 

\noindent El proyecto \textbf{NoCKt Metal} se encuentra actualmente en el 
siguiente estado:

\begin{itemize}
\item Motor de movimiento
  \begin{itemize}
  \item Creación de un mapa a partir de tiles de 30x30
  \item Movimiento de un personaje
  \item Scrolling del mapa si fuera necesario al salir de un rango
  \item Colisiones en ciertos tiles
  \item Implementación de los distintos eventos lanzados por el usuario
  \end{itemize}

\item Motor de combate
  \begin{itemize}
  \item Implementación básica del nucleo del sistema de combate.
  \item Lectura y escritura de los distintos elementos mediante
    ficheros XML.
  \item Pequeña interfaz por consola para probar el motor.
  \item Organización en grupos, uso de inventarios y habilidades.
  \end{itemize}
\end{itemize}

\noindent Por tanto, analizamos que las siguientes tareas serán:

\begin{itemize}
\item Motor de movimiento
  \begin{itemize}
  \item Inclusión de menús.
  \item Interacción con elementos del mapa (objetos ó PNJs).
  \item Añadir mapas ``serios''
  \item Diálogos.
  \item Añadir lectura y escritura usando XML.
  \end{itemize}

\item Motor de combate
  \begin{itemize}
  \item Implementar las funciones \textit{huir()} y \textit{defensa()}.
  \item Estudiar IA de los enemigos.
  \item Añadir a las clases que se necesite guardar, la ruta al fichero
    donde se guardará, como un atributo de la clase.
  \item Interfaz gráfica usando SDL
  \item Realizar las operaciones de postcombate
  \item Añadir más elementos (objetos,habilidades, personajes...)
  \item Eliminar la clase \texttt{Atributos} dada la poca entidad por
    si misma que tiene en el sistema.
  \end{itemize}
\end{itemize}

\noindent Sin especificar concretamente que tareas realizaremos cada uno de
los 3 miembros, \textit{a grosso modo} la organización por orden alfabético
es la siguiente:

\begin{description}
\item[Noelia: ] Realización de la interfaz para los combates. Notar que
  los menús se reutilizarán al menos en parte para el motor de movimiento.
\item[Pablo: ] Continuar el desarrollo del nucleo del motor de combate. 
  Ficheros XML en el motor de movimiento.
\item[Rosa: ] Continuar el desarrollo del motor de movimiento. Añadir quizás
elementos como más mapas, diálogos y demás.
\end{description}

\end{document}
